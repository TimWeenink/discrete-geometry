\documentclass[11pt]{amsart}

\usepackage{a4wide}
\usepackage{paralist}
\usepackage{url}
\usepackage{nopageno}
\usepackage{bbm}
\usepackage[normalem]{ulem}

\newcommand{\cA}{\mathcal{A}}
\newcommand{\cS}{\mathcal{S}}
\newcommand{\R}{\mathbbm{R}}
\newcommand{\Z}{\mathbbm{Z}}
\DeclareMathOperator{\conv}{conv}
\DeclareMathOperator{\cols}{cols}
\DeclareMathOperator{\Sl}{Sl}

\begin{document}
\begin{center}
\textbf{\sffamily
   Discrete and Algorithmic Geometry }

\medskip
   Julian Pfeifle,
   UPC, 2014 \mbox{}
\end{center}

\bigskip

\begin{center}
  \textbf{\sffamily Sheet 2}

\bigskip
 due on \sout{Monday, November 17, 2014} \\
 Friday, November 21, 2014

\end{center}

\bigskip
\bigskip
\bigskip

\section*{Writing}

\begin{enumerate}
\setlength{\itemsep}{2ex}
\item Let $P\subset\R^d$, $Q\subset\R^e$ be two non-empty polytopes. Prove that the set of faces of the cartesian product polytope $P\times Q=\{(p,q)\in\R^{d+e}:p\in P,\; q\in Q\}$ exactly equals $\{F\times G: F\text{ is face of }P, \;G\text{ is face of }Q\}$. Conclude that
\[
    f_k(P\times Q)
    \ = \
    \sum_{%\substack{
      i+j=k,\;i,j\ge0}f_i(P) f_j(Q)
    \qquad
    \text{for } k\ge0,
\]
and use this formula to calculate the entire $f$-vector of the permutahedron 
\[
  P_n 
  \ = \ 
  \conv\big\{\big(\pi(1),\pi(2),\dots,\pi(n)\big)^\top:\pi\in S_n\big\}.
\]

\item A \emph{lattice polytope} is the convex hull of finitely many
  vertices with integer coordinates. Two lattice polytopes
  $P,Q\subset\R^d$ are \emph{lattice equivalent} or \emph{lattice
    isomorphic} if there exist $A\in\Z^{d\times d}$ with
  $\det(A)=\pm1$ and $b\in\Z^d$ such that $Q=f(P)$ for the affine map
  $f(x)=Ax+b$. (How is this different from demanding that $P$ and $Q$
  be $\Sl_d(\Z)$-equivalent?)

  \begin{enumerate}
  \item Prove that the \emph{modular group} $\Sl_2(\Z)$ is generated by the elements $S=
    \big(\begin{smallmatrix}
      0 & -1 \\
      1 & 0
    \end{smallmatrix}\big)$ and $T=
    \big(\begin{smallmatrix}
      1 & 1 \\
      0 & 1
    \end{smallmatrix}\big)$, i.e., any $A\in\Sl_2(\Z)$ is expressible as a product of matrices $S$~and~$T$.

  \item Interpret the preceding result geometrically, and use it to
    classify the lattice polygons with (i) no, (ii) exactly one
    strictly interior lattice point up to lattice equivalence.
  \end{enumerate}
\end{enumerate}

\bigskip
\bigskip
\section*{Software}

\begin{enumerate}
\setlength{\itemsep}{2ex}
\setcounter{enumi}{2}
\item Explain the difference between a public/private key pair
  for~\texttt{ssh} and a public/private key pair
  for~\texttt{gpg}. Gather information about the recommended keysizes,
  and explain briefly the advantages and disadvantages of the
  \texttt{gpg}~software, making special mention of the latest
  versions.  Then place a public/private \texttt{gpg}(!)~keypair of
  the recommended size into \texttt{2014/public\_keys}~. Again, don't
  forget to commit and push your changes, and issue a pull request.

\item Using your favorite software environment, write a function~$\chi$ that
  checks if two 3-dimensional lattice tetrahedra are lattice
  equivalent.
%
  Your function should take two $4\times4$ matrices $P,Q$ as input,
  and output a $4\times 4$ matrix~$A$.
%
  The columns of the input matrices $P$~and~$Q$ are to be interpreted
  as the homogeneous coordinates of the vertices of the respective
  lattice tetrahedra. The output $A$ is either $0$ (if $\widetilde P = \conv\cols P$
  is not lattice isomorphic to $\widetilde Q$), or encodes a
  $\Z$-affine map that sends $\widetilde P$ to $\widetilde Q$.

  In particular, your function should correctly reproduce the calculation

  \[
  \chi\left(
  \begin{bmatrix}
    1 & 1 & 1 & 1 \\
    0 & 1 & 0 & 2 \\
    0 & 0 & 1 & 3 \\
    0 & 0 & 0 & 5
  \end{bmatrix}
  ,\;
  \begin{bmatrix}
    1 & 1 & 1 & 1 \\
    0 & 0 & 1 & 2 \\
    0 & 1 & 0 & 3 \\
    0 & 0 & 0 & 5
  \end{bmatrix}
  \right)
  \ = \ 
  I_4.
  \]

\medskip
  Provide a rudimentary testing facility and testsuite for your
  function. You should be able to input the two matrices from two
  plain text files, where each row encodes a vertex of the
  tetrahedron, and write the output matrix to a third file. For
  instance, the matrix $P$ above would be encoded as
\begin{verbatim}
1 0 0 0
1 1 0 0
1 0 1 0
1 2 3 5
\end{verbatim}

  Also prepare a function \texttt{test\_chi} that reads a file containing lines of the form 
\begin{verbatim}
matrix_P1.txt matrix_P2.txt result_P1_P2.text
matrix_P3.txt matrix_P3.txt result_P3_P4.text
\end{verbatim}
  and executes the tests with the corresponding filenames. When you
  are done, your instructor will provide you with more examples to test your function on!

  \medskip Create a directory of the form
  \texttt{2014/exercises/sheet2/your\_name\_here/code} and put all
  relevant source code in there. Don't forget a README so that other
  people (i.e., you in the future) can figure out how your function
  works! And, as always, don't forget the pull request.
\end{enumerate}


\end{document}
