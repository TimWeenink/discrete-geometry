\documentclass[11pt]{amsart}

\usepackage{a4wide}
\usepackage{paralist}
\usepackage{url}
\usepackage{bbm}
\usepackage{nopageno}

\newcommand{\cA}{\mathcal{A}}
\newcommand{\cS}{\mathcal{S}}
\DeclareMathOperator{\conv}{conv}
\DeclareMathOperator{\New}{New}
\DeclareMathOperator{\area}{area}
\newcommand{\RR}{\mathbbm{R}}
\newcommand{\CC}{\mathbbm{C}}

\begin{document}
\begin{center}
\textbf{\sffamily
   Discrete and Algorithmic Geometry }

\medskip
   Julian Pfeifle,
   UPC, 2016
\end{center}

\bigskip

\begin{center}
  \textbf{\sffamily Sheet 1}

\bigskip
 due on Monday, November 13, 2016

\end{center}

\bigskip
\bigskip
\bigskip

\section*{Reading}

\begin{enumerate}
\setlength{\itemsep}{2ex}
\item Read Lectures 0,1,2 from Ziegler's \emph{Lectures on Polytopes}.

\item Read Sections 5.1, 5.2, 5.3 from Matou\v sek's \emph{Lectures on
    Discrete Geometry}.

\end{enumerate}

\bigskip
\bigskip
\section*{Writing}

The \emph{Minkowski sum} of two polytopes $P,Q\subset\RR^d$ is the set
\[
   P + Q
   \ = \
   \big\{p+q : p\in P,\,q\in Q\}.
\]
% As our running example, we consider the polygons
% \[
%    P
%    \ = \
%    \conv\big\{
%      (0,0),\;(1,0),\;(1,1),\;(0,1)
%    \big\},
%    \qquad
%    Q
%    \ = \
%    \conv\big\{
%      (0,0),\;(2,1),\;(1,2)
%    \big\}.
% \]
The \emph{Newton polytope} of a polynomial $f(x)=\sum_{i=1}^m c_i \, x_1^{a_{i1}}x_2^{a_{i2}}\cdots x_d^{a_{id}} \in \RR[x_1,x_2,\dots,x_d]$ is 
\[
   \New(f)
   \ = \
   \conv\big\{(a_{11},a_{12},\dots,a_{1d}),\dots,(a_{m1},a_{m2},\dots,a_{md})\big\}
   \ \subset \
   \RR^d.
\]
\begin{enumerate}[(a)]
\item Prove that $P+Q\subset\RR^d$ is a convex polytope. How does $P+Q$ change when we translate $P$~and~$Q$ in~$\RR^d$?

\item How does the Minkowski sum $\New(f)+\New(g)$ (a geometric object) relate to the algebraic objects $f$~and~$g$?  

\item Let $f(x,y)=c_0 + c_1 x + c_2 xy + c_3 y$ and $g(x,y) = d_0 + d_1 x^2y + d_2 xy^2$ be two polynomials with $c_i\in\RR$ and $d_i\in\RR$ for all~$i$, and let $P=\New(f)$ and $Q=\New(g)$ be their Newton polygons. Calculate the \emph{mixed area}
\[
  M(P,Q)
  \ = \
  \area(P+Q) - \area(P) - \area(Q).
\]

\item Subdivide $P+Q$ into five pieces: a translate of $P$, a translate of $Q$, and three parallelograms, and relate $M(P,Q)$ to this subdivision.

\medskip
\emph{Bernstein's Theorem} asserts that for any generic polynomials $f$ and $g$ in two variables, the number of solutions in $(\CC^\times)^2$ of the system $f(x,y)=0$, $g(x,y)=0$ equals the mixed area $M\big(\New(f),\New(g)\big)$. Here $\CC^\times = \CC\setminus\{0\}$. In fact, this theorem applies to \emph{Laurent polynomials}, which means that the exponents can be arbitrary integers, possibly negative. 

\medskip
\item Use Bernstein's Theorem to conclude the geometric fact that for any \emph{lattice polygons} $P,Q$ (i.e., all vertices have integer coordinates), the mixed area $M(P,Q)$ is an integer.

\item Use Bernstein's Theorem to conclude \emph{Bezout's Theorem}: Two general plane algebraic curves of degree $d$ respectively~$e$ meet in $d\cdot e$~points. (\emph{Hint:} Consider the Newton polytopes of two general bivariate polynomials of degree $d$ and $e$, respectively.) How does this relate to the polynomials $f$ and $g$ in parts (c) and (d)? (Be sure you understand ---or look up--- the difference between a \emph{generic} and a \emph{general} polynomial.)
\end{enumerate}
\end{document}
 